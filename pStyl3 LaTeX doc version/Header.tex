\documentclass[a4paper,english,titlepage]{scrartcl}
\usepackage{geometry}																			% Layout of the document
\geometry{a4paper,																				% Format
					left=10mm,																			% Distance to left screen border
					right=10mm,																			% Distance to right screen border
					top=15mm,																				% Distance to top screen border
					bottom=15mm,																		% Distance to bottom screen border
					footskip=8mm																		% Setting for the page number
					}
\usepackage{lmodern}																			% Smoothing of fonts in the .pdf document
\usepackage{helvet}
\renewcommand{\familydefault}{\sfdefault}
\usepackage[T1]{fontenc}																	% Definition of the character set
\usepackage[latin1]{inputenc}															% Package for detection of umlauts
\usepackage[english]{babel}																% Language
\usepackage[automark]{scrpage2}														% german: http://fff2.at/drupal/content/latex-paket-scrpage2     english: https://www.ctan.org/pkg/scrpage2?lang=en
\pagestyle{scrheadings}
\clearscrheadfoot
%\ifoot[]{\author}																				% i = inner, c = center, o = outer; head/foot
\ohead{\headmark}
\cfoot{- \pagemark\ -}
\usepackage{caption}
\captionsetup[figure]{labelformat=empty}									% Redefines the caption setup of the figures environment in the beamer class.
\usepackage{graphicx}																			% Import image files: .pdf .jpg .png .tif
\usepackage[export]{adjustbox}
\usepackage{scrpage2}
\usepackage{amsmath}																			% Mathematical formulas
\usepackage{amsfonts}																			% i.a. Quantity symbols
\usepackage{amssymb}																			% i.a. Symbol "Empty Set" \varnothing
\usepackage{float}																				% Force Image Position [H]
\usepackage{latexsym}																			% Package for LaTeX-Symbols
\usepackage{pdflscape}																		% Presentation and format of the PDF document
\usepackage{xcolor}																				% Colors
\usepackage{colortbl}
%\usepackage{cite}																				% Citation
\frenchspacing																						% Turns off the extra space, that LaTeX normally inserts after a punctuation mark.
\usepackage{tabularx}																			% Package for defining the width of columns in tables  \begin{tabularx}{\textwidth}{XX} \end{tabularx}
\usepackage{longtable}																		% Package for larger tables across multiple pages
\usepackage{booktabs}																			% Package for thick, vertical separators  \toprule, \midrule, \bottomrule
\usepackage{soul}																					% Package for underlines  \ul  -->  e.g. \setul{1ex}{0.8ex}
\usepackage{fancybox}																			% Packages for frames
\usepackage[strict]{changepage}														% For adjustwidth environment
\usepackage{framed}																				% For formal definitions
\usepackage{listings}																			% Code boxes
\usepackage{enumitem}
\usepackage{url}																					% Display of Internet addresses
\usepackage[hidelinks]{hyperref}													% Advanced referencing options. This package must be loaded last!  [pdfpagelayout=SinglePage]



% Color - Table of Contents
\definecolor{LinkColor}{rgb}{0,0,0.5}											% Numbers between 0 and 1
\hypersetup{%
colorlinks=true,%
breaklinks=true,
linkcolor=LinkColor,%
%citecolor=LinkColor,%
%filecolor=LinkColor,%
%menucolor=LinkColor,%
%pagecolor=LinkColor,%
urlcolor=LinkColor,
%hyperindex=true
}


% Specify Underlines  using  \setul{depth}{thickness}
\setul{0.225ex}{0.125ex}


% Create new commands to color all links in LinkColor and underline them
\newcommand{\xhyperref}[2]{\hyperref[#1]{\color{LinkColor}\setulcolor{LinkColor}\ul{#2}}}
\newcommand{\xnameref}[1]{\color{LinkColor}\underline{\nameref{#1}}\color{black}}
\newcommand{\xhref}[2]{\href{#1}{\color{LinkColor}\setulcolor{LinkColor}\ul{#2}}}


% Color Boxes in Images
%Turquoise
%#D0EFFF
%RGB = 208 - 239 - 255
%\color[rgb]{0.813,0.934,1.0}

%Red
%#FF0000
%RGB = 255 - 0 - 0
%\color[rgb]{1.0,0.0,0.0}

%Blue
%#325FA0
%RGB = 50 - 95 - 160
%\color[rgb]{0.195,0.371,0.625}


% Design of Code Boxes
% http://en.wikibooks.org/wiki/LaTeX/Source_Code_Listings#Style_definition
\lstdefinestyle{customc}{
  belowcaptionskip=1\baselineskip,
  breaklines=true,
  frame=L,
  xleftmargin=\parindent,
  language=C,
  showstringspaces=false,
  basicstyle=\footnotesize\ttfamily,
  keywordstyle=\bfseries\color{green!40!black},
  commentstyle=\itshape\color{purple!40!black},
  identifierstyle=\color{blue},
  stringstyle=\color{orange},
}
\lstdefinestyle{customasm}{
  belowcaptionskip=1\baselineskip,
  frame=L,
  xleftmargin=\parindent,
  language=[x86masm]Assembler,
  basicstyle=\footnotesize\ttfamily,
  commentstyle=\itshape\color{purple!40!black},
}
\lstset{escapechar=@,style=customc}


% Inclusion of \paragraph and \subparagraph in den TOC
\setcounter{tocdepth}{2}			% Inclusion in the table of contents
\setcounter{secnumdepth}{4}		% Deepen numbering  ({5} for subparagraph)

% Distance of itemize-listpoints
\newcommand{\CustomItemGap}{-0.5em}




% environment derived from framed.sty: see leftbar environment definition
% http://www.jevon.org/wiki/Fancy_Quotation_Boxes_in_Latex
% see  \usepackage[strict]{changepage}   and   \usepackage{framed}
%
\definecolor{bluebar}{rgb}{0.129,0.161,0.549}
\definecolor{formalshade}{rgb}{0.95,0.95,1}

\newenvironment{formal}{%
  \def\FrameCommand{%
    \hspace{1pt}%
    {\color{bluebar}\vrule width 2pt}%
    {\color{formalshade}\vrule width 4pt}%
    \colorbox{formalshade}%
  }%
  \MakeFramed{\advance\hsize-\width\FrameRestore}%
  \noindent\hspace{-4.55pt}% disable indenting first paragraph
  \begin{adjustwidth}{}{7pt}%
  \vspace{2pt}\vspace{2pt}%
}
{%
  \vspace{2pt}\end{adjustwidth}\endMakeFramed%
}


% Change TOC title
\addto\captionsenglish{
  \renewcommand{\contentsname}
    {Table of Contents}
}


% Specify the used file extension for all images within the document
\newcommand{\CFE}{jpg}		% CFE = Color File Extension